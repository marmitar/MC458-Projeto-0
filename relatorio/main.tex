\documentclass[a4paper, 14pt]{extarticle}

\usepackage[portuguese]{babel}
\usepackage[utf8]{inputenc}
\usepackage[T1]{fontenc}
\usepackage[margin=2.2cm]{geometry}

\usepackage{float, pgf, caption, subcaption}

% \input{secoes}
% \input{teorema}
\input{simbolos}

% math display skip
\newcommand{\reducemathskip}[1][0.5em]{%
    \setlength{\abovedisplayskip}{1pt}%
    \setlength{\belowdisplayskip}{#1}%
    \setlength{\abovedisplayshortskip}{#1}%
    \setlength{\belowdisplayshortskip}{#1}%
}

% url linking problems
\def\url#1{\href{#1}{\texttt{#1}}}
% vermelho
\def\red#1{\textcolor{red}{#1}}

\def\lmref#1{\thmref[lema ]{#1}}

\usepackage{xparse}

\newtheorem*{hypothesis}{Hipótese}
\newtheorem*{hypothesisf}{Hipótese Fortalecida}

\NewDocumentCommand{\seq}{ s m O{n} O{\in\natural} }
    {\IfBooleanTF{#1}
        {\ensuremath{\left({#2}_{#3}\right)}}
        {\ensuremath{\left({#2}_{#3}\right)_{{#3}{#4}}}}}


\title{\vspace{-2.5cm}Projeto de Algoritmo com Implementação nº 0 \\ \normalsize MC458 - 2s2020}
\author{Tiago de Paula Alves \\ \small 187679}
\date{}

\begin{document}
\maketitle

\section{Implementação}

    O algoritmo de cálculo de estabilidade foi feito considerando o móbile como uma árvore especializada. No caso, cada móbile tem dois objetos $O_e$ e $O_d$, com distâncias $D_e$ e $D_d$ do sustentáculo. Cada objeto $O$ pode ser um outro móbile ou um peso simples, sendo o peso $P_O$ conhecido apenas de objetos simples. Dessa forma, um móbile $M$ funciona como uma árvore não-vazia, sendo cada submódule um nó interno da árvore e objetos simples são as suas folhas.

    Assim, cada peso simples $S$ tem uma medida de peso $P_S$ fixa, recebida da entrada padrão. Os móbiles $M$, no entanto, têm peso dependente de seus objetos $E$ e $D$. Então, podemos definir a função peso $P: \text{Objeto} \to \real$, que calcula o peso total de um objeto $O$, como:
    \[
        P(O) = \begin{cases}
            P_O & \text{se $O$ é um peso simples} \\
            P(E) + P(D) & \text{se $O$ é um móbile com objetos $E$ e $D$}
        \end{cases}
    \]

    Com isso, um móbile $M$ com objetos $E$ e $D$, de pesos $P_E$ e $P_D$ e às distâncias $D_E$ e $D_D$, está em equilíbrio se seus submóbiles estão em equilíbrio e se $P_E \cdot D_E = P_D \cdot D_D$. Expandindo para objetos em geral, podemos tratar um peso simples como sempre em equilíbrio, de forma que $M$ estará em equilíbrio se e somente se $E$ está em equilíbrio, $D$ está em equilíbrio e $P(E) \cdot D_E = P(D) \cdot D_D$.

\section{Algoritmo}

    O algoritmo pode ser deduzido a partir de uma indução forte, como comumente acontece para árvores binárias. Nesse caso, a primeira hipótese de indução seria:

    \begin{hypothesis}
        Dado um móbile com $n$ objetos, conseguimos vericar se ele está em equilíbrio.
    \end{hypothesis}

    No entanto, esse tratamento funciona para objetos com peso definidos ou já conhecidos. Então, podemos fortalecer a hipótese com o cálculo da função peso $P$ definida anteriormente. Além disso, podemos usar a definição de equilíbrio que abrange os dois tipos de objetos, não só móbiles, fazendo uma hipótese que encobre os dois casos.

    \begin{hypothesisf}
        Dado um objeto $O$ com $n$ subobjetos, podemos calcular seu peso $P(O)$ e vericar se ele está em equilíbrio.
    \end{hypothesisf}

\section{Alg}

    \begin{proof}
        Suponha um inteiro $n \geq 1$ tal que para todo objeto $O$ composto de $1 \leq k < n$ massas podemos descobrir seu peso total $P(O)$ e se ele está em equilíbrio. Suponha ainda um objeto $O$ com $n$ subobjetos.

        ~

        Caso 1: $O$ é um peso simples, ou seja, $n = 1$. Logo, o peso de é conhecido, então seja $P_O$ esse peso. Portanto, $P(O) = P_O$ e, por definição, $O$ está em equilíbrio.

        ~

        Caso 2: $O$ é um móbile. Seja $E$ o objeto peso na ponta esquerda de $O$, à uma distância $D_E$ do sustentáculo, e $D$ o objeto na direita, com distância $D_D$.

        Note que $D$ deve ter pelo $k_D \geq 1$ massas, que é no caso em que $D$ é um peso simples. Portanto, $E$ terá $1 \leq k_E = n - k_D \leq n - 1 < n$ massas. Logo, pela hipótese indutiva, podemos calcular $P(E)$ e sabemos se $E$ está em equilíbrio. Da mesma forma, $1 \leq k_D < n$. Então, podemos conseguir $P(D)$ e saber se $D$ está em equilíbrio.

        Assim, se $E$ e $D$ estão em equilíbrio e $P(E) * D_E = P(D) * D_D$, então $O$ também está em equilíbrio. Caso alguma dessas condições não seja atendida, $O$ não está em equilíbrio. Portanto, teremos que $P(O) = P(E) + P(D)$ e podemos dizer se $O$ está em equilíbrio.
    \end{proof}


    % \begin{proof}
    %     Suponha um inteiro $n \geq 2$ tal que para todo mobile $M$ composto de $2 \leq k < n$ pesos simples podemos descobrir seu peso $P(M)$ e se ele está em equilíbrio. Suponha ainda um móbile $M$ com $n$ pesos.

    %     Caso 1: $M$ é um móbilo simples. Logo, temos dois pesos simples $e$ e $d$, com pesos $P_e$ e $P_d$ e
    % \end{proof}

\end{document}
